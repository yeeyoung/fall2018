\documentclass[]{article}
\usepackage{lmodern}
\usepackage{amssymb,amsmath}
\usepackage{ifxetex,ifluatex}
\usepackage{fixltx2e} % provides \textsubscript
\ifnum 0\ifxetex 1\fi\ifluatex 1\fi=0 % if pdftex
  \usepackage[T1]{fontenc}
  \usepackage[utf8]{inputenc}
\else % if luatex or xelatex
  \ifxetex
    \usepackage{mathspec}
  \else
    \usepackage{fontspec}
  \fi
  \defaultfontfeatures{Ligatures=TeX,Scale=MatchLowercase}
\fi
% use upquote if available, for straight quotes in verbatim environments
\IfFileExists{upquote.sty}{\usepackage{upquote}}{}
% use microtype if available
\IfFileExists{microtype.sty}{%
\usepackage{microtype}
\UseMicrotypeSet[protrusion]{basicmath} % disable protrusion for tt fonts
}{}
\usepackage[margin=1in]{geometry}
\usepackage{hyperref}
\hypersetup{unicode=true,
            pdftitle={STAT545 HW3},
            pdfauthor={Yi Yang},
            pdfborder={0 0 0},
            breaklinks=true}
\urlstyle{same}  % don't use monospace font for urls
\usepackage{color}
\usepackage{fancyvrb}
\newcommand{\VerbBar}{|}
\newcommand{\VERB}{\Verb[commandchars=\\\{\}]}
\DefineVerbatimEnvironment{Highlighting}{Verbatim}{commandchars=\\\{\}}
% Add ',fontsize=\small' for more characters per line
\usepackage{framed}
\definecolor{shadecolor}{RGB}{248,248,248}
\newenvironment{Shaded}{\begin{snugshade}}{\end{snugshade}}
\newcommand{\KeywordTok}[1]{\textcolor[rgb]{0.13,0.29,0.53}{\textbf{#1}}}
\newcommand{\DataTypeTok}[1]{\textcolor[rgb]{0.13,0.29,0.53}{#1}}
\newcommand{\DecValTok}[1]{\textcolor[rgb]{0.00,0.00,0.81}{#1}}
\newcommand{\BaseNTok}[1]{\textcolor[rgb]{0.00,0.00,0.81}{#1}}
\newcommand{\FloatTok}[1]{\textcolor[rgb]{0.00,0.00,0.81}{#1}}
\newcommand{\ConstantTok}[1]{\textcolor[rgb]{0.00,0.00,0.00}{#1}}
\newcommand{\CharTok}[1]{\textcolor[rgb]{0.31,0.60,0.02}{#1}}
\newcommand{\SpecialCharTok}[1]{\textcolor[rgb]{0.00,0.00,0.00}{#1}}
\newcommand{\StringTok}[1]{\textcolor[rgb]{0.31,0.60,0.02}{#1}}
\newcommand{\VerbatimStringTok}[1]{\textcolor[rgb]{0.31,0.60,0.02}{#1}}
\newcommand{\SpecialStringTok}[1]{\textcolor[rgb]{0.31,0.60,0.02}{#1}}
\newcommand{\ImportTok}[1]{#1}
\newcommand{\CommentTok}[1]{\textcolor[rgb]{0.56,0.35,0.01}{\textit{#1}}}
\newcommand{\DocumentationTok}[1]{\textcolor[rgb]{0.56,0.35,0.01}{\textbf{\textit{#1}}}}
\newcommand{\AnnotationTok}[1]{\textcolor[rgb]{0.56,0.35,0.01}{\textbf{\textit{#1}}}}
\newcommand{\CommentVarTok}[1]{\textcolor[rgb]{0.56,0.35,0.01}{\textbf{\textit{#1}}}}
\newcommand{\OtherTok}[1]{\textcolor[rgb]{0.56,0.35,0.01}{#1}}
\newcommand{\FunctionTok}[1]{\textcolor[rgb]{0.00,0.00,0.00}{#1}}
\newcommand{\VariableTok}[1]{\textcolor[rgb]{0.00,0.00,0.00}{#1}}
\newcommand{\ControlFlowTok}[1]{\textcolor[rgb]{0.13,0.29,0.53}{\textbf{#1}}}
\newcommand{\OperatorTok}[1]{\textcolor[rgb]{0.81,0.36,0.00}{\textbf{#1}}}
\newcommand{\BuiltInTok}[1]{#1}
\newcommand{\ExtensionTok}[1]{#1}
\newcommand{\PreprocessorTok}[1]{\textcolor[rgb]{0.56,0.35,0.01}{\textit{#1}}}
\newcommand{\AttributeTok}[1]{\textcolor[rgb]{0.77,0.63,0.00}{#1}}
\newcommand{\RegionMarkerTok}[1]{#1}
\newcommand{\InformationTok}[1]{\textcolor[rgb]{0.56,0.35,0.01}{\textbf{\textit{#1}}}}
\newcommand{\WarningTok}[1]{\textcolor[rgb]{0.56,0.35,0.01}{\textbf{\textit{#1}}}}
\newcommand{\AlertTok}[1]{\textcolor[rgb]{0.94,0.16,0.16}{#1}}
\newcommand{\ErrorTok}[1]{\textcolor[rgb]{0.64,0.00,0.00}{\textbf{#1}}}
\newcommand{\NormalTok}[1]{#1}
\usepackage{graphicx,grffile}
\makeatletter
\def\maxwidth{\ifdim\Gin@nat@width>\linewidth\linewidth\else\Gin@nat@width\fi}
\def\maxheight{\ifdim\Gin@nat@height>\textheight\textheight\else\Gin@nat@height\fi}
\makeatother
% Scale images if necessary, so that they will not overflow the page
% margins by default, and it is still possible to overwrite the defaults
% using explicit options in \includegraphics[width, height, ...]{}
\setkeys{Gin}{width=\maxwidth,height=\maxheight,keepaspectratio}
\IfFileExists{parskip.sty}{%
\usepackage{parskip}
}{% else
\setlength{\parindent}{0pt}
\setlength{\parskip}{6pt plus 2pt minus 1pt}
}
\setlength{\emergencystretch}{3em}  % prevent overfull lines
\providecommand{\tightlist}{%
  \setlength{\itemsep}{0pt}\setlength{\parskip}{0pt}}
\setcounter{secnumdepth}{0}
% Redefines (sub)paragraphs to behave more like sections
\ifx\paragraph\undefined\else
\let\oldparagraph\paragraph
\renewcommand{\paragraph}[1]{\oldparagraph{#1}\mbox{}}
\fi
\ifx\subparagraph\undefined\else
\let\oldsubparagraph\subparagraph
\renewcommand{\subparagraph}[1]{\oldsubparagraph{#1}\mbox{}}
\fi

%%% Use protect on footnotes to avoid problems with footnotes in titles
\let\rmarkdownfootnote\footnote%
\def\footnote{\protect\rmarkdownfootnote}

%%% Change title format to be more compact
\usepackage{titling}

% Create subtitle command for use in maketitle
\newcommand{\subtitle}[1]{
  \posttitle{
    \begin{center}\large#1\end{center}
    }
}

\setlength{\droptitle}{-2em}

  \title{STAT545 HW3}
    \pretitle{\vspace{\droptitle}\centering\huge}
  \posttitle{\par}
    \author{Yi Yang}
    \preauthor{\centering\large\emph}
  \postauthor{\par}
      \predate{\centering\large\emph}
  \postdate{\par}
    \date{9/26/2018}


\begin{document}
\maketitle

\section{1. Problem 1: The K-means
algorithm.}\label{problem-1-the-k-means-algorithm.}

The MNIST dataset is a dataset of \(28\times 28\) images of hand-written
digits. Download it from \url{http://yann.lecun.com/exdb/mnist/} (you
only really need the training images and labels though). To read these
images in R, use the script from
\url{https://gist.github.com/brendano/39760}. Make sure you understand
this. Note that the \(\mathrm{show\_digit}\) command displays a
particular digit.

\begin{enumerate}
\def\labelenumi{\arabic{enumi}.}
\item
  Since the dataset is quite large, restrict yourself to the first 1000
  training images, and their labels. Store these as variables called
  \(\mathrm{digits}\) and \(\mathrm{labels}\). \(\mathrm{digits}\)
  should be a \(1000\times 784\) matrix (or its transpose). Include R
  code.
\item
  Write a function \(\mathrm{my\_kmeans}\) to perform a k-means
  clustering of the 1000 images of digits. Use Euclidean distance as
  your distance measure between images (which can be viewed as vectors
  in a 784 dimensional space). Your function should take 3 arguments,
  the matrix \(\mathrm{digits}\), the number of clusters \(K\) and the
  number of initializations \(N\). You code should consist of 3 nested
  loops. The outermost (from \(1\) to \(N\)) cycles over random cluster
  initializations (i.e.~you will call k-means \(N\) times with different
  initializations). The second loop (this could be a for or while loop)
  is the actual k-means algorithm for that initialization, and cycles
  over the iterations of k-means. Inside this are the actual iterations
  of k-means. Each iteration can have 2 successive loops from 1 to
  \(K\): the first assigns observations to each cluster and the second
  recalculates the means of each cluster. These should not require
  further loops. (You will probably encounter empty clusters. It is
  possible to deal with these in clever ways, but here it is sufficient
  to assign empty clusters a random mean (just like you initialized
  them)). Since your initializations are random, make your results
  repeatable by using the \(\mathrm{set.seed()}\) command at the
  beginning (you can also make the seed value a fourth argument). Your
  function should return:
\end{enumerate}

\begin{enumerate}
\def\labelenumi{(\alph{enumi})}
\tightlist
\item
  the cluster parameters and cluster assignments for the best solution
\item
  the sequence of values of the loss-function over k-means iterations
  for the best solution (this should be non-increasing) (recall from the
  slides that the k-means loss function is the sum of the squared
  distances of observations from their assigned means)
\item
  The set of \(N\) terminal loss-function values for all
  initializations.
\end{enumerate}

Do not hardcode the number of images or their size. Include R code.

\begin{enumerate}
\def\labelenumi{\arabic{enumi}.}
\setcounter{enumi}{2}
\item
  Explain briefly what stopping criteria you used (i.e.~the details of
  the second loop).
\item
  Run your code on the 1000 digits for \(K = 5, 10, 20\). Set \(N\) to a
  largish number e.g.~25 (if this takes too long, use a smaller
  number).For each setting of \(K\), plot the cluster means (using
  \(\mathrm{show\_digit}\)) as well as the evolution of the
  loss-function for the best solution (you can use a semi-log plot if
  that is clearer). You do not have to print the other values returned
  by the function e.g.~the cluster assignments, or the values of the
  cluster means etc., just plots is sufficient
\item
  For each setting of \(K\), plot the distribution of terminal loss
  function values (using \(\mathrm{ggplot}\)'s
  \(\mathrm{geom\_density}()\)).
\item
  Explain briefly how you might choose the number of clusters \(K\).
\end{enumerate}

\subsubsection{Bonus for an extra 20
points:}\label{bonus-for-an-extra-20-points}

\begin{enumerate}
\def\labelenumi{\arabic{enumi}.}
\setcounter{enumi}{6}
\item
  Modify your code to do k-medoids. You only need to change one part of
  your previous code viz. the part that calculates the cluster prototype
  given cluster assignments. The cleanest way to do this is to define a
  function called \(\mathrm{get\_prototype}\) that takes a set of
  observations and returns the prototype. For k-means this function just
  returns the mean of the observations. Note that the mean can also be
  defined as \[\mu = \arg \min_x \sum_{i = 1}^{|D|}(x - d_i)^2\] Here
  \(D = (d_1, \dots, D_{|D|})\) is the set of images input to
  \(\mathrm{get\_prototype}\), and the mean need not be part of this
  set. For k-medoids, the prototype is defined as
  \[\mu = \arg \min_{x\in D} \sum_{i = 1}^{|D|}(x - d_i)^2\] In other
  words it finds an element in \(D\) that minimizes the sum-squared
  distance. Include R code for your implementation of
  \(\mathrm{get\_prototype}\) for k-mediods. You can use as many for
  loops as you want, but the simplest is to loop over each observation
  assigned to that cluster, calculate the sum-squared distance when that
  is set as the prototype, and return the best choice.
\item
  Run k-medoids for \(K = 5, 10, 20\). Since this might take longer, you
  can use smaller values of \(N\) as well as fewer images (e.g.~just 100
  digits), but report what numbers you used. For each choice of \(K\),
  show the cluster prototypes. Comment on the quality of the cluster
  prototypes, as well as the value of the loss function vs k-means.
\end{enumerate}

\textbf{Solution:}

\begin{enumerate}
\def\labelenumi{\arabic{enumi}.}
\tightlist
\item
  read the first 1000 digits and their labels
\end{enumerate}

\begin{Shaded}
\begin{Highlighting}[]
\CommentTok{# Credit to https://gist.github.com/brendano/39760}
\CommentTok{# Load the MNIST digit recognition dataset into R}

\NormalTok{load_image_file <-}\StringTok{ }\ControlFlowTok{function}\NormalTok{(filename) \{}
\NormalTok{  ret =}\StringTok{ }\KeywordTok{list}\NormalTok{()}
\NormalTok{  f =}\StringTok{ }\KeywordTok{file}\NormalTok{(filename,}\StringTok{'rb'}\NormalTok{)}
  \KeywordTok{readBin}\NormalTok{(f,}\StringTok{'integer'}\NormalTok{,}\DataTypeTok{n=}\DecValTok{1}\NormalTok{,}\DataTypeTok{size=}\DecValTok{4}\NormalTok{,}\DataTypeTok{endian=}\StringTok{'big'}\NormalTok{)}
\NormalTok{  ret}\OperatorTok{$}\NormalTok{n =}\StringTok{ }\KeywordTok{readBin}\NormalTok{(f,}\StringTok{'integer'}\NormalTok{,}\DataTypeTok{n=}\DecValTok{1}\NormalTok{,}\DataTypeTok{size=}\DecValTok{4}\NormalTok{,}\DataTypeTok{endian=}\StringTok{'big'}\NormalTok{)}
\NormalTok{  nrow =}\StringTok{ }\KeywordTok{readBin}\NormalTok{(f,}\StringTok{'integer'}\NormalTok{,}\DataTypeTok{n=}\DecValTok{1}\NormalTok{,}\DataTypeTok{size=}\DecValTok{4}\NormalTok{,}\DataTypeTok{endian=}\StringTok{'big'}\NormalTok{)}
\NormalTok{  ncol =}\StringTok{ }\KeywordTok{readBin}\NormalTok{(f,}\StringTok{'integer'}\NormalTok{,}\DataTypeTok{n=}\DecValTok{1}\NormalTok{,}\DataTypeTok{size=}\DecValTok{4}\NormalTok{,}\DataTypeTok{endian=}\StringTok{'big'}\NormalTok{)}
\NormalTok{  x =}\StringTok{ }\KeywordTok{readBin}\NormalTok{(f,}\StringTok{'integer'}\NormalTok{,}\DataTypeTok{n=}\NormalTok{ret}\OperatorTok{$}\NormalTok{n}\OperatorTok{*}\NormalTok{nrow}\OperatorTok{*}\NormalTok{ncol,}\DataTypeTok{size=}\DecValTok{1}\NormalTok{,}\DataTypeTok{signed=}\NormalTok{F)}
\NormalTok{  ret}\OperatorTok{$}\NormalTok{x =}\StringTok{ }\KeywordTok{matrix}\NormalTok{(x, }\DataTypeTok{ncol=}\NormalTok{nrow}\OperatorTok{*}\NormalTok{ncol, }\DataTypeTok{byrow=}\NormalTok{T)}
  \KeywordTok{close}\NormalTok{(f)}
\NormalTok{  ret}
\NormalTok{\}}
\NormalTok{load_label_file <-}\StringTok{ }\ControlFlowTok{function}\NormalTok{(filename) \{}
\NormalTok{  f =}\StringTok{ }\KeywordTok{file}\NormalTok{(filename,}\StringTok{'rb'}\NormalTok{)}
  \KeywordTok{readBin}\NormalTok{(f,}\StringTok{'integer'}\NormalTok{,}\DataTypeTok{n=}\DecValTok{1}\NormalTok{,}\DataTypeTok{size=}\DecValTok{4}\NormalTok{,}\DataTypeTok{endian=}\StringTok{'big'}\NormalTok{)}
\NormalTok{  n =}\StringTok{ }\KeywordTok{readBin}\NormalTok{(f,}\StringTok{'integer'}\NormalTok{,}\DataTypeTok{n=}\DecValTok{1}\NormalTok{,}\DataTypeTok{size=}\DecValTok{4}\NormalTok{,}\DataTypeTok{endian=}\StringTok{'big'}\NormalTok{)}
\NormalTok{  y =}\StringTok{ }\KeywordTok{readBin}\NormalTok{(f,}\StringTok{'integer'}\NormalTok{,}\DataTypeTok{n=}\NormalTok{n,}\DataTypeTok{size=}\DecValTok{1}\NormalTok{,}\DataTypeTok{signed=}\NormalTok{F)}
  \KeywordTok{close}\NormalTok{(f)}
\NormalTok{  y}
\NormalTok{\}}
 
\NormalTok{show_digit <-}\StringTok{ }\ControlFlowTok{function}\NormalTok{(arr784, }\DataTypeTok{col=}\KeywordTok{gray}\NormalTok{(}\DecValTok{12}\OperatorTok{:}\DecValTok{1}\OperatorTok{/}\DecValTok{12}\NormalTok{), ...) \{}
  \KeywordTok{image}\NormalTok{(}\KeywordTok{matrix}\NormalTok{(arr784, }\DataTypeTok{nrow=}\DecValTok{28}\NormalTok{)[,}\DecValTok{28}\OperatorTok{:}\DecValTok{1}\NormalTok{], }\DataTypeTok{col=}\NormalTok{col,}\DataTypeTok{axes=}\NormalTok{F, ...)}
\NormalTok{\}}

\CommentTok{# load the first 1000 digits and their labels}
\NormalTok{digits.list <-}\StringTok{ }\KeywordTok{load_image_file}\NormalTok{(}\StringTok{'./data/train-images-idx3-ubyte'}\NormalTok{)}
\NormalTok{digits <-}\StringTok{ }\NormalTok{digits.list}\OperatorTok{$}\NormalTok{x[}\DecValTok{1}\OperatorTok{:}\DecValTok{1000}\NormalTok{,]}
\NormalTok{digits <-}\StringTok{ }\NormalTok{digits }\OperatorTok{/}\StringTok{ }\FloatTok{255.0} \CommentTok{# normalize pixels}
\NormalTok{labels <-}\StringTok{ }\KeywordTok{load_label_file}\NormalTok{(}\StringTok{'./data/train-labels-idx1-ubyte'}\NormalTok{)}
\KeywordTok{dim}\NormalTok{(digits)}
\end{Highlighting}
\end{Shaded}

\begin{verbatim}
## [1] 1000  784
\end{verbatim}

\begin{Shaded}
\begin{Highlighting}[]
\CommentTok{# snippets to test the function of show_digits}
\KeywordTok{show_digit}\NormalTok{(digits[}\DecValTok{12}\NormalTok{,])}
\end{Highlighting}
\end{Shaded}

\includegraphics{HW3_files/figure-latex/unnamed-chunk-2-1.pdf}

\begin{Shaded}
\begin{Highlighting}[]
\NormalTok{labels[}\DecValTok{12}\NormalTok{]}
\end{Highlighting}
\end{Shaded}

\begin{verbatim}
## [1] 5
\end{verbatim}

\begin{Shaded}
\begin{Highlighting}[]
\KeywordTok{class}\NormalTok{(labels[}\DecValTok{12}\NormalTok{])}
\end{Highlighting}
\end{Shaded}

\begin{verbatim}
## [1] "integer"
\end{verbatim}

\begin{enumerate}
\def\labelenumi{\arabic{enumi}.}
\setcounter{enumi}{1}
\tightlist
\item
  K-means function is given by
\end{enumerate}

\begin{Shaded}
\begin{Highlighting}[]
\CommentTok{# define helper functions}
\CommentTok{# get distance between a single digit and all the centroids of clusters}
\NormalTok{getDist <-}\StringTok{ }\ControlFlowTok{function}\NormalTok{(single.digits, centroids)\{}
\NormalTok{  dist <-}\StringTok{ }\KeywordTok{colSums}\NormalTok{((}\KeywordTok{t}\NormalTok{(centroids) }\OperatorTok{-}\StringTok{ }\NormalTok{single.digits)}\OperatorTok{^}\DecValTok{2}\NormalTok{)}
  \KeywordTok{return}\NormalTok{(dist)}
\NormalTok{\}}
\CommentTok{# update the centroid for a single cluster by calculating the mean of digits in this cluster}
\NormalTok{single.centroid <-}\StringTok{ }\ControlFlowTok{function}\NormalTok{(ric, all.digits)\{}
\NormalTok{  num.digits.in.cluster <-}\StringTok{ }\KeywordTok{sum}\NormalTok{(ric)}
  \ControlFlowTok{if}\NormalTok{ (num.digits.in.cluster }\OperatorTok{>}\StringTok{ }\DecValTok{0}\NormalTok{)\{}
    \KeywordTok{return}\NormalTok{((ric }\OperatorTok\StringTok{ }\NormalTok{all.digits) }\OperatorTok{/}\StringTok{ }\NormalTok{num.digits.in.cluster )}
\NormalTok{  \}}
  \ControlFlowTok{else}\NormalTok{\{}
    \KeywordTok{return}\NormalTok{(all.digits[}\KeywordTok{sample}\NormalTok{(}\KeywordTok{nrow}\NormalTok{(all.digits),}\DecValTok{1}\NormalTok{),])}
\NormalTok{  \}}
\NormalTok{\}}
\CommentTok{# K-means algorithm starts here }
\CommentTok{# K is num of clusters, N is num of different initializations}
\NormalTok{my_kmeans <-}\StringTok{ }\ControlFlowTok{function}\NormalTok{(all.digits, K, N)\{}
  \KeywordTok{set.seed}\NormalTok{(}\DecValTok{16}\NormalTok{)}
\NormalTok{  best.cluster.parameters <-}\StringTok{ }\OtherTok{NULL} \CommentTok{# centroids}
\NormalTok{  best.cluster.assignments <-}\StringTok{ }\OtherTok{NULL} \CommentTok{# ric}
\NormalTok{  best.loss.seq <-}\StringTok{ }\KeywordTok{c}\NormalTok{()}
\NormalTok{  terminal.losses <-}\StringTok{ }\KeywordTok{c}\NormalTok{()}
\NormalTok{  best.terminal.loss <-}\StringTok{ }\OtherTok{Inf}
  \ControlFlowTok{for}\NormalTok{ (i }\ControlFlowTok{in} \KeywordTok{seq}\NormalTok{(}\DecValTok{1}\NormalTok{,N))\{}
\NormalTok{    centroids <-}\StringTok{ }\NormalTok{all.digits[}\KeywordTok{sample}\NormalTok{(}\KeywordTok{nrow}\NormalTok{(all.digits),K), ]}
\NormalTok{    dists <-}\StringTok{ }\KeywordTok{apply}\NormalTok{(all.digits,}\DecValTok{1}\NormalTok{,getDist,}\DataTypeTok{centroids=}\NormalTok{centroids)}
\NormalTok{    dists <-}\StringTok{ }\KeywordTok{t}\NormalTok{(dists)}
\NormalTok{    loss <-}\StringTok{ }\KeywordTok{sum}\NormalTok{(}\KeywordTok{apply}\NormalTok{(dists, }\DecValTok{1}\NormalTok{, min))}
\NormalTok{    last.loss <-}\StringTok{ }\NormalTok{loss }\OperatorTok{+}\StringTok{ }\DecValTok{1}
\NormalTok{    loss.seq <-}\StringTok{ }\KeywordTok{c}\NormalTok{(loss)}
    \ControlFlowTok{while}\NormalTok{ (}\KeywordTok{abs}\NormalTok{(last.loss }\OperatorTok{-}\StringTok{ }\NormalTok{loss) }\OperatorTok{>=}\StringTok{ }\FloatTok{1e-4}\NormalTok{) \{}
\NormalTok{      rics <-}\StringTok{ }\KeywordTok{apply}\NormalTok{(dists,}\DecValTok{1}\NormalTok{,}\ControlFlowTok{function}\NormalTok{(d) }\KeywordTok{ifelse}\NormalTok{(d}\OperatorTok{==}\KeywordTok{min}\NormalTok{(d),}\DecValTok{1}\NormalTok{,}\DecValTok{0}\NormalTok{))}
\NormalTok{      rics <-}\StringTok{ }\KeywordTok{t}\NormalTok{(rics)}
\NormalTok{      centroids <-}\StringTok{ }\KeywordTok{apply}\NormalTok{(rics, }\DecValTok{2}\NormalTok{, single.centroid, }\DataTypeTok{all.digits=}\NormalTok{all.digits)}
\NormalTok{      dists <-}\StringTok{ }\KeywordTok{apply}\NormalTok{(all.digits, }\DecValTok{1}\NormalTok{, getDist, }\DataTypeTok{centroids=}\KeywordTok{t}\NormalTok{(centroids))}
\NormalTok{      dists <-}\StringTok{ }\KeywordTok{t}\NormalTok{(dists)}
\NormalTok{      last.loss <-}\StringTok{ }\NormalTok{loss}
\NormalTok{      loss <-}\StringTok{ }\KeywordTok{sum}\NormalTok{(}\KeywordTok{apply}\NormalTok{(dists, }\DecValTok{1}\NormalTok{, min))}
\NormalTok{      loss.seq <-}\StringTok{ }\KeywordTok{c}\NormalTok{(loss.seq,loss)}
\NormalTok{    \}}
\NormalTok{    terminal.losses <-}\StringTok{ }\KeywordTok{c}\NormalTok{(terminal.losses,loss)}
    \ControlFlowTok{if}\NormalTok{ (best.terminal.loss }\OperatorTok{>}\StringTok{ }\NormalTok{loss)\{}
\NormalTok{      best.terminal.loss <-}\StringTok{ }\NormalTok{loss}
\NormalTok{      best.loss.seq <-}\StringTok{ }\NormalTok{loss.seq}
\NormalTok{      best.cluster.assignments <-}\StringTok{ }\NormalTok{rics}
\NormalTok{      best.cluster.parameters <-}\StringTok{ }\KeywordTok{t}\NormalTok{(centroids)}
\NormalTok{    \}}
\NormalTok{  \}}
  \KeywordTok{return}\NormalTok{(}\KeywordTok{list}\NormalTok{(}\DataTypeTok{best.cluster.parameters=}\NormalTok{best.cluster.parameters, }\DataTypeTok{best.cluster.assignments=}\NormalTok{best.cluster.assignments, }\DataTypeTok{best.loss.seq=}\NormalTok{best.loss.seq, }\DataTypeTok{terminal.losses=}\NormalTok{terminal.losses))}
\NormalTok{\}}
\end{Highlighting}
\end{Shaded}

\begin{enumerate}
\def\labelenumi{\arabic{enumi}.}
\setcounter{enumi}{2}
\tightlist
\item
  The stopping criterion is set to be that the difference of the loss
  function in the last iteration and the following iteration is smaller
  than \(1\times 10^{-4}\).
\item
  The cluster means are plotted below, as well as the best loss
  sequence.
\end{enumerate}

\begin{Shaded}
\begin{Highlighting}[]
\NormalTok{K <-}\StringTok{ }\KeywordTok{c}\NormalTok{(}\DecValTok{5}\NormalTok{,}\DecValTok{10}\NormalTok{,}\DecValTok{20}\NormalTok{)}
\NormalTok{N <-}\StringTok{ }\DecValTok{15}
\NormalTok{results <-}\StringTok{ }\KeywordTok{list}\NormalTok{()}
\ControlFlowTok{for}\NormalTok{ (k }\ControlFlowTok{in}\NormalTok{ K)\{}
\NormalTok{  results <-}\StringTok{ }\KeywordTok{append}\NormalTok{(results, }\KeywordTok{list}\NormalTok{(}\KeywordTok{my_kmeans}\NormalTok{(digits,k,N)))}
\NormalTok{\}}
\end{Highlighting}
\end{Shaded}

\begin{itemize}
\tightlist
\item
  Case \(K = 5\)
\end{itemize}

\begin{Shaded}
\begin{Highlighting}[]
\ControlFlowTok{for}\NormalTok{ (id }\ControlFlowTok{in} \KeywordTok{seq}\NormalTok{(}\DecValTok{1}\NormalTok{,}\DecValTok{5}\NormalTok{))\{}
  \KeywordTok{show_digit}\NormalTok{(results[[}\DecValTok{1}\NormalTok{]]}\OperatorTok{$}\NormalTok{best.cluster.parameters[id,])}
\NormalTok{\}}
\end{Highlighting}
\end{Shaded}

\includegraphics{HW3_files/figure-latex/unnamed-chunk-5-1.pdf}
\includegraphics{HW3_files/figure-latex/unnamed-chunk-5-2.pdf}
\includegraphics{HW3_files/figure-latex/unnamed-chunk-5-3.pdf}
\includegraphics{HW3_files/figure-latex/unnamed-chunk-5-4.pdf}
\includegraphics{HW3_files/figure-latex/unnamed-chunk-5-5.pdf} * Case
\(K = 10\)

\begin{Shaded}
\begin{Highlighting}[]
\ControlFlowTok{for}\NormalTok{ (id }\ControlFlowTok{in} \KeywordTok{seq}\NormalTok{(}\DecValTok{1}\NormalTok{,}\DecValTok{10}\NormalTok{))\{}
  \KeywordTok{show_digit}\NormalTok{(results[[}\DecValTok{2}\NormalTok{]]}\OperatorTok{$}\NormalTok{best.cluster.parameters[id,])}
\NormalTok{\}}
\end{Highlighting}
\end{Shaded}

\includegraphics{HW3_files/figure-latex/unnamed-chunk-6-1.pdf}
\includegraphics{HW3_files/figure-latex/unnamed-chunk-6-2.pdf}
\includegraphics{HW3_files/figure-latex/unnamed-chunk-6-3.pdf}
\includegraphics{HW3_files/figure-latex/unnamed-chunk-6-4.pdf}
\includegraphics{HW3_files/figure-latex/unnamed-chunk-6-5.pdf}
\includegraphics{HW3_files/figure-latex/unnamed-chunk-6-6.pdf}
\includegraphics{HW3_files/figure-latex/unnamed-chunk-6-7.pdf}
\includegraphics{HW3_files/figure-latex/unnamed-chunk-6-8.pdf}
\includegraphics{HW3_files/figure-latex/unnamed-chunk-6-9.pdf}
\includegraphics{HW3_files/figure-latex/unnamed-chunk-6-10.pdf} * Case
\(K = 20\)

\begin{Shaded}
\begin{Highlighting}[]
\ControlFlowTok{for}\NormalTok{ (id }\ControlFlowTok{in} \KeywordTok{seq}\NormalTok{(}\DecValTok{1}\NormalTok{,}\DecValTok{20}\NormalTok{))\{}
  \KeywordTok{show_digit}\NormalTok{(results[[}\DecValTok{3}\NormalTok{]]}\OperatorTok{$}\NormalTok{best.cluster.parameters[id,])}
\NormalTok{\}}
\end{Highlighting}
\end{Shaded}

\includegraphics{HW3_files/figure-latex/unnamed-chunk-7-1.pdf}
\includegraphics{HW3_files/figure-latex/unnamed-chunk-7-2.pdf}
\includegraphics{HW3_files/figure-latex/unnamed-chunk-7-3.pdf}
\includegraphics{HW3_files/figure-latex/unnamed-chunk-7-4.pdf}
\includegraphics{HW3_files/figure-latex/unnamed-chunk-7-5.pdf}
\includegraphics{HW3_files/figure-latex/unnamed-chunk-7-6.pdf}
\includegraphics{HW3_files/figure-latex/unnamed-chunk-7-7.pdf}
\includegraphics{HW3_files/figure-latex/unnamed-chunk-7-8.pdf}
\includegraphics{HW3_files/figure-latex/unnamed-chunk-7-9.pdf}
\includegraphics{HW3_files/figure-latex/unnamed-chunk-7-10.pdf}
\includegraphics{HW3_files/figure-latex/unnamed-chunk-7-11.pdf}
\includegraphics{HW3_files/figure-latex/unnamed-chunk-7-12.pdf}
\includegraphics{HW3_files/figure-latex/unnamed-chunk-7-13.pdf}
\includegraphics{HW3_files/figure-latex/unnamed-chunk-7-14.pdf}
\includegraphics{HW3_files/figure-latex/unnamed-chunk-7-15.pdf}
\includegraphics{HW3_files/figure-latex/unnamed-chunk-7-16.pdf}
\includegraphics{HW3_files/figure-latex/unnamed-chunk-7-17.pdf}
\includegraphics{HW3_files/figure-latex/unnamed-chunk-7-18.pdf}
\includegraphics{HW3_files/figure-latex/unnamed-chunk-7-19.pdf}
\includegraphics{HW3_files/figure-latex/unnamed-chunk-7-20.pdf} *
Evolution for the best solution in case \(K = 5\)

\begin{Shaded}
\begin{Highlighting}[]
\KeywordTok{library}\NormalTok{(ggplot2)}
\NormalTok{best.loss.seq.df <-}\StringTok{ }\KeywordTok{data.frame}\NormalTok{(}\DataTypeTok{loss=}\NormalTok{results[[}\DecValTok{1}\NormalTok{]]}\OperatorTok{$}\NormalTok{best.loss.seq,}\DataTypeTok{iter=}\KeywordTok{seq}\NormalTok{(}\DecValTok{1}\NormalTok{,}\KeywordTok{length}\NormalTok{(results[[}\DecValTok{1}\NormalTok{]]}\OperatorTok{$}\NormalTok{best.loss.seq)))}
\KeywordTok{ggplot}\NormalTok{(best.loss.seq.df,}\KeywordTok{aes}\NormalTok{(}\DataTypeTok{x=}\NormalTok{iter,}\DataTypeTok{y=}\NormalTok{loss)) }\OperatorTok{+}\StringTok{ }
\StringTok{  }\KeywordTok{geom_line}\NormalTok{() }\OperatorTok{+}\StringTok{ }\KeywordTok{geom_point}\NormalTok{() }\OperatorTok{+}\StringTok{ }\KeywordTok{ggtitle}\NormalTok{(}\DataTypeTok{label =} \StringTok{'K = 5'}\NormalTok{)}
\end{Highlighting}
\end{Shaded}

\includegraphics{HW3_files/figure-latex/unnamed-chunk-8-1.pdf} *
Evolution for the best solution in case \(K = 10\)

\begin{Shaded}
\begin{Highlighting}[]
\NormalTok{best.loss.seq.df <-}\StringTok{ }\KeywordTok{data.frame}\NormalTok{(}\DataTypeTok{loss=}\NormalTok{results[[}\DecValTok{2}\NormalTok{]]}\OperatorTok{$}\NormalTok{best.loss.seq,}\DataTypeTok{iter=}\KeywordTok{seq}\NormalTok{(}\DecValTok{1}\NormalTok{,}\KeywordTok{length}\NormalTok{(results[[}\DecValTok{2}\NormalTok{]]}\OperatorTok{$}\NormalTok{best.loss.seq)))}
\KeywordTok{ggplot}\NormalTok{(best.loss.seq.df,}\KeywordTok{aes}\NormalTok{(}\DataTypeTok{x=}\NormalTok{iter,}\DataTypeTok{y=}\NormalTok{loss)) }\OperatorTok{+}\StringTok{ }
\StringTok{  }\KeywordTok{geom_line}\NormalTok{() }\OperatorTok{+}\StringTok{ }\KeywordTok{geom_point}\NormalTok{() }\OperatorTok{+}\StringTok{ }\KeywordTok{ggtitle}\NormalTok{(}\DataTypeTok{label =} \StringTok{'K = 10'}\NormalTok{)}
\end{Highlighting}
\end{Shaded}

\includegraphics{HW3_files/figure-latex/unnamed-chunk-9-1.pdf} *
Evolution for the best solution in case \(K = 20\)

\begin{Shaded}
\begin{Highlighting}[]
\NormalTok{best.loss.seq.df <-}\StringTok{ }\KeywordTok{data.frame}\NormalTok{(}\DataTypeTok{loss=}\NormalTok{results[[}\DecValTok{3}\NormalTok{]]}\OperatorTok{$}\NormalTok{best.loss.seq,}\DataTypeTok{iter=}\KeywordTok{seq}\NormalTok{(}\DecValTok{1}\NormalTok{,}\KeywordTok{length}\NormalTok{(results[[}\DecValTok{3}\NormalTok{]]}\OperatorTok{$}\NormalTok{best.loss.seq)))}
\KeywordTok{ggplot}\NormalTok{(best.loss.seq.df,}\KeywordTok{aes}\NormalTok{(}\DataTypeTok{x=}\NormalTok{iter,}\DataTypeTok{y=}\NormalTok{loss)) }\OperatorTok{+}\StringTok{ }
\StringTok{  }\KeywordTok{geom_line}\NormalTok{() }\OperatorTok{+}\StringTok{ }\KeywordTok{geom_point}\NormalTok{() }\OperatorTok{+}\StringTok{ }\KeywordTok{ggtitle}\NormalTok{(}\DataTypeTok{label =} \StringTok{'K = 20'}\NormalTok{)}
\end{Highlighting}
\end{Shaded}

\includegraphics{HW3_files/figure-latex/unnamed-chunk-10-1.pdf}

\begin{enumerate}
\def\labelenumi{\arabic{enumi}.}
\setcounter{enumi}{4}
\tightlist
\item
  The distribution of terminal loss function values are given by
\end{enumerate}

\begin{itemize}
\tightlist
\item
  Case \(K = 5\)
\end{itemize}

\begin{Shaded}
\begin{Highlighting}[]
\NormalTok{terminal.losses.df <-}\StringTok{ }\KeywordTok{data.frame}\NormalTok{(}\DataTypeTok{terminal.losses=}\NormalTok{results[[}\DecValTok{1}\NormalTok{]]}\OperatorTok{$}\NormalTok{terminal.losses)}
\KeywordTok{ggplot}\NormalTok{(terminal.losses.df,}\KeywordTok{aes}\NormalTok{(}\DataTypeTok{x=}\NormalTok{terminal.losses)) }\OperatorTok{+}\StringTok{ }
\StringTok{  }\KeywordTok{geom_density}\NormalTok{() }\OperatorTok{+}\StringTok{ }\KeywordTok{ggtitle}\NormalTok{(}\DataTypeTok{label =} \StringTok{'K = 5'}\NormalTok{)}
\end{Highlighting}
\end{Shaded}

\includegraphics{HW3_files/figure-latex/unnamed-chunk-11-1.pdf} * Case
\(K = 10\)

\begin{Shaded}
\begin{Highlighting}[]
\NormalTok{terminal.losses.df <-}\StringTok{ }\KeywordTok{data.frame}\NormalTok{(}\DataTypeTok{terminal.losses=}\NormalTok{results[[}\DecValTok{2}\NormalTok{]]}\OperatorTok{$}\NormalTok{terminal.losses)}
\KeywordTok{ggplot}\NormalTok{(terminal.losses.df,}\KeywordTok{aes}\NormalTok{(}\DataTypeTok{x=}\NormalTok{terminal.losses)) }\OperatorTok{+}\StringTok{ }
\StringTok{  }\KeywordTok{geom_density}\NormalTok{() }\OperatorTok{+}\StringTok{ }\KeywordTok{ggtitle}\NormalTok{(}\DataTypeTok{label =} \StringTok{'K = 10'}\NormalTok{)}
\end{Highlighting}
\end{Shaded}

\includegraphics{HW3_files/figure-latex/unnamed-chunk-12-1.pdf} * Case
\(K = 20\)

\begin{Shaded}
\begin{Highlighting}[]
\NormalTok{terminal.losses.df <-}\StringTok{ }\KeywordTok{data.frame}\NormalTok{(}\DataTypeTok{terminal.losses=}\NormalTok{results[[}\DecValTok{3}\NormalTok{]]}\OperatorTok{$}\NormalTok{terminal.losses)}
\KeywordTok{ggplot}\NormalTok{(terminal.losses.df,}\KeywordTok{aes}\NormalTok{(}\DataTypeTok{x=}\NormalTok{terminal.losses)) }\OperatorTok{+}\StringTok{ }
\StringTok{  }\KeywordTok{geom_density}\NormalTok{() }\OperatorTok{+}\StringTok{ }\KeywordTok{ggtitle}\NormalTok{(}\DataTypeTok{label =} \StringTok{'K = 20'}\NormalTok{)}
\end{Highlighting}
\end{Shaded}

\includegraphics{HW3_files/figure-latex/unnamed-chunk-13-1.pdf}

\begin{enumerate}
\def\labelenumi{\arabic{enumi}.}
\setcounter{enumi}{5}
\tightlist
\item
  Plot best terminal losses with a set of different \(K\) values, we
  show the curve will look like an elbow. Picking the \(K\) value around
  the elbow point is appropriate.
\end{enumerate}

\subsubsection{Bonus for an extra 20
points:}\label{bonus-for-an-extra-20-points-1}

\begin{enumerate}
\def\labelenumi{\arabic{enumi}.}
\setcounter{enumi}{6}
\tightlist
\item
  For K-medoids method, we only need to modify function
  \(\textrm{single.centroid}\).
\end{enumerate}

\begin{Shaded}
\begin{Highlighting}[]
\CommentTok{# get_prototype function is a substitute to function single.centroid}
\NormalTok{get_prototype <-}\StringTok{ }\ControlFlowTok{function}\NormalTok{(ric, all.digits)\{}
\NormalTok{  digits.in.cluster <-}\StringTok{ }\NormalTok{all.digits[ric}\OperatorTok{==}\DecValTok{1}\NormalTok{,]}
\NormalTok{  smallest.dist <-}\StringTok{ }\OtherTok{Inf}
\NormalTok{  closest.digit <-}\StringTok{ }\NormalTok{digits.in.cluster[}\DecValTok{1}\NormalTok{,]}
  \ControlFlowTok{if}\NormalTok{ (}\KeywordTok{nrow}\NormalTok{(digits.in.cluster) }\OperatorTok{==}\StringTok{ }\DecValTok{0}\NormalTok{)\{}
    \KeywordTok{return}\NormalTok{(all.digits[}\KeywordTok{sample}\NormalTok{(}\KeywordTok{nrow}\NormalTok{(all.digits),}\DecValTok{1}\NormalTok{), ])}
\NormalTok{  \}}
  \ControlFlowTok{for}\NormalTok{ (i }\ControlFlowTok{in} \KeywordTok{seq}\NormalTok{(}\DecValTok{1}\NormalTok{,}\KeywordTok{nrow}\NormalTok{(digits.in.cluster)))\{}
\NormalTok{    digit <-}\StringTok{ }\NormalTok{digits.in.cluster[i, ]}
\NormalTok{    dist.to.others <-}\StringTok{ }\KeywordTok{sum}\NormalTok{(}\KeywordTok{getDist}\NormalTok{(digit,digits.in.cluster))}
    \ControlFlowTok{if}\NormalTok{ (dist.to.others }\OperatorTok{<}\StringTok{ }\NormalTok{smallest.dist)\{}
\NormalTok{      closest.digit <-}\StringTok{ }\NormalTok{digit}
\NormalTok{      smallest.dist <-}\StringTok{ }\NormalTok{dist.to.others}
\NormalTok{    \}}
\NormalTok{  \}}
  \KeywordTok{return}\NormalTok{(closest.digit)}
\NormalTok{\}}
\CommentTok{# define function my_kmedoids}
\NormalTok{my_kmedoids <-}\StringTok{ }\ControlFlowTok{function}\NormalTok{(all.digits, K, N)\{}
  \KeywordTok{set.seed}\NormalTok{(}\DecValTok{26}\NormalTok{)}
\NormalTok{  best.cluster.parameters <-}\StringTok{ }\OtherTok{NULL} \CommentTok{# centroids}
\NormalTok{  best.cluster.assignments <-}\StringTok{ }\OtherTok{NULL} \CommentTok{# ric}
\NormalTok{  best.loss.seq <-}\StringTok{ }\KeywordTok{c}\NormalTok{()}
\NormalTok{  terminal.losses <-}\StringTok{ }\KeywordTok{c}\NormalTok{()}
\NormalTok{  best.terminal.loss <-}\StringTok{ }\OtherTok{Inf}
  \ControlFlowTok{for}\NormalTok{ (i }\ControlFlowTok{in} \KeywordTok{seq}\NormalTok{(}\DecValTok{1}\NormalTok{,N))\{}
\NormalTok{    centroids <-}\StringTok{ }\NormalTok{all.digits[}\KeywordTok{sample}\NormalTok{(}\KeywordTok{nrow}\NormalTok{(all.digits),K), ]}
\NormalTok{    dists <-}\StringTok{ }\KeywordTok{apply}\NormalTok{(all.digits,}\DecValTok{1}\NormalTok{,getDist,}\DataTypeTok{centroids=}\NormalTok{centroids)}
\NormalTok{    dists <-}\StringTok{ }\KeywordTok{t}\NormalTok{(dists)}
\NormalTok{    loss <-}\StringTok{ }\KeywordTok{sum}\NormalTok{(}\KeywordTok{apply}\NormalTok{(dists, }\DecValTok{1}\NormalTok{, min))}
\NormalTok{    last.loss <-}\StringTok{ }\NormalTok{loss }\OperatorTok{+}\StringTok{ }\DecValTok{1}
\NormalTok{    loss.seq <-}\StringTok{ }\KeywordTok{c}\NormalTok{(loss)}
    \ControlFlowTok{while}\NormalTok{ (}\KeywordTok{abs}\NormalTok{(last.loss }\OperatorTok{-}\StringTok{ }\NormalTok{loss) }\OperatorTok{>=}\StringTok{ }\FloatTok{1e-4}\NormalTok{) \{}
\NormalTok{      rics <-}\StringTok{ }\KeywordTok{apply}\NormalTok{(dists,}\DecValTok{1}\NormalTok{,}\ControlFlowTok{function}\NormalTok{(d) }\KeywordTok{ifelse}\NormalTok{(d}\OperatorTok{==}\KeywordTok{min}\NormalTok{(d),}\DecValTok{1}\NormalTok{,}\DecValTok{0}\NormalTok{))}
\NormalTok{      rics <-}\StringTok{ }\KeywordTok{t}\NormalTok{(rics)}
\NormalTok{      centroids <-}\StringTok{ }\KeywordTok{apply}\NormalTok{(rics, }\DecValTok{2}\NormalTok{, get_prototype, }\DataTypeTok{all.digits=}\NormalTok{all.digits)}
\NormalTok{      dists <-}\StringTok{ }\KeywordTok{apply}\NormalTok{(all.digits, }\DecValTok{1}\NormalTok{, getDist, }\DataTypeTok{centroids=}\KeywordTok{t}\NormalTok{(centroids))}
\NormalTok{      dists <-}\StringTok{ }\KeywordTok{t}\NormalTok{(dists)}
\NormalTok{      last.loss <-}\StringTok{ }\NormalTok{loss}
\NormalTok{      loss <-}\StringTok{ }\KeywordTok{sum}\NormalTok{(}\KeywordTok{apply}\NormalTok{(dists, }\DecValTok{1}\NormalTok{, min))}
\NormalTok{      loss.seq <-}\StringTok{ }\KeywordTok{c}\NormalTok{(loss.seq,loss)}
\NormalTok{    \}}
\NormalTok{    terminal.losses <-}\StringTok{ }\KeywordTok{c}\NormalTok{(terminal.losses,loss)}
    \ControlFlowTok{if}\NormalTok{ (best.terminal.loss }\OperatorTok{>}\StringTok{ }\NormalTok{loss)\{}
\NormalTok{      best.terminal.loss <-}\StringTok{ }\NormalTok{loss}
\NormalTok{      best.loss.seq <-}\StringTok{ }\NormalTok{loss.seq}
\NormalTok{      best.cluster.assignments <-}\StringTok{ }\NormalTok{rics}
\NormalTok{      best.cluster.parameters <-}\StringTok{ }\KeywordTok{t}\NormalTok{(centroids)}
\NormalTok{    \}}
\NormalTok{  \}}
  \KeywordTok{return}\NormalTok{(}\KeywordTok{list}\NormalTok{(}\DataTypeTok{best.cluster.parameters=}\NormalTok{best.cluster.parameters, }\DataTypeTok{best.cluster.assignments=}\NormalTok{best.cluster.assignments, }\DataTypeTok{best.loss.seq=}\NormalTok{best.loss.seq, }\DataTypeTok{terminal.losses=}\NormalTok{terminal.losses))}
\NormalTok{\}}
\end{Highlighting}
\end{Shaded}

\begin{enumerate}
\def\labelenumi{\arabic{enumi}.}
\setcounter{enumi}{7}
\tightlist
\item
  Let us take \(N = 10\) and first \(100\) digits. The cluster
  prototypes are clearer than those obtained from K-means method. The
  loss function values are relatively higher than K-means method.
\end{enumerate}

\begin{Shaded}
\begin{Highlighting}[]
\NormalTok{digits.smaller <-}\StringTok{ }\NormalTok{digits[}\DecValTok{1}\OperatorTok{:}\DecValTok{100}\NormalTok{, ]}
\NormalTok{K <-}\StringTok{ }\KeywordTok{c}\NormalTok{(}\DecValTok{5}\NormalTok{,}\DecValTok{10}\NormalTok{,}\DecValTok{20}\NormalTok{)}
\NormalTok{N <-}\StringTok{ }\DecValTok{10}
\NormalTok{results <-}\StringTok{ }\KeywordTok{list}\NormalTok{()}
\ControlFlowTok{for}\NormalTok{ (k }\ControlFlowTok{in}\NormalTok{ K)\{}
\NormalTok{  results <-}\StringTok{ }\KeywordTok{append}\NormalTok{(results, }\KeywordTok{list}\NormalTok{(}\KeywordTok{my_kmedoids}\NormalTok{(digits,k,N)))}
\NormalTok{\}}
\end{Highlighting}
\end{Shaded}

\begin{itemize}
\tightlist
\item
  Case \(K = 5\)
\end{itemize}

\begin{Shaded}
\begin{Highlighting}[]
\ControlFlowTok{for}\NormalTok{ (id }\ControlFlowTok{in} \KeywordTok{seq}\NormalTok{(}\DecValTok{1}\NormalTok{,}\DecValTok{5}\NormalTok{))\{}
  \KeywordTok{show_digit}\NormalTok{(results[[}\DecValTok{1}\NormalTok{]]}\OperatorTok{$}\NormalTok{best.cluster.parameters[id,])}
\NormalTok{\}}
\end{Highlighting}
\end{Shaded}

\includegraphics{HW3_files/figure-latex/unnamed-chunk-16-1.pdf}
\includegraphics{HW3_files/figure-latex/unnamed-chunk-16-2.pdf}
\includegraphics{HW3_files/figure-latex/unnamed-chunk-16-3.pdf}
\includegraphics{HW3_files/figure-latex/unnamed-chunk-16-4.pdf}
\includegraphics{HW3_files/figure-latex/unnamed-chunk-16-5.pdf} * Case
\(K = 10\)

\begin{Shaded}
\begin{Highlighting}[]
\ControlFlowTok{for}\NormalTok{ (id }\ControlFlowTok{in} \KeywordTok{seq}\NormalTok{(}\DecValTok{1}\NormalTok{,}\DecValTok{10}\NormalTok{))\{}
  \KeywordTok{show_digit}\NormalTok{(results[[}\DecValTok{2}\NormalTok{]]}\OperatorTok{$}\NormalTok{best.cluster.parameters[id,])}
\NormalTok{\}}
\end{Highlighting}
\end{Shaded}

\includegraphics{HW3_files/figure-latex/unnamed-chunk-17-1.pdf}
\includegraphics{HW3_files/figure-latex/unnamed-chunk-17-2.pdf}
\includegraphics{HW3_files/figure-latex/unnamed-chunk-17-3.pdf}
\includegraphics{HW3_files/figure-latex/unnamed-chunk-17-4.pdf}
\includegraphics{HW3_files/figure-latex/unnamed-chunk-17-5.pdf}
\includegraphics{HW3_files/figure-latex/unnamed-chunk-17-6.pdf}
\includegraphics{HW3_files/figure-latex/unnamed-chunk-17-7.pdf}
\includegraphics{HW3_files/figure-latex/unnamed-chunk-17-8.pdf}
\includegraphics{HW3_files/figure-latex/unnamed-chunk-17-9.pdf}
\includegraphics{HW3_files/figure-latex/unnamed-chunk-17-10.pdf} * Case
\(K = 20\)

\begin{Shaded}
\begin{Highlighting}[]
\ControlFlowTok{for}\NormalTok{ (id }\ControlFlowTok{in} \KeywordTok{seq}\NormalTok{(}\DecValTok{1}\NormalTok{,}\DecValTok{20}\NormalTok{))\{}
  \KeywordTok{show_digit}\NormalTok{(results[[}\DecValTok{3}\NormalTok{]]}\OperatorTok{$}\NormalTok{best.cluster.parameters[id,])}
\NormalTok{\}}
\end{Highlighting}
\end{Shaded}

\includegraphics{HW3_files/figure-latex/unnamed-chunk-18-1.pdf}
\includegraphics{HW3_files/figure-latex/unnamed-chunk-18-2.pdf}
\includegraphics{HW3_files/figure-latex/unnamed-chunk-18-3.pdf}
\includegraphics{HW3_files/figure-latex/unnamed-chunk-18-4.pdf}
\includegraphics{HW3_files/figure-latex/unnamed-chunk-18-5.pdf}
\includegraphics{HW3_files/figure-latex/unnamed-chunk-18-6.pdf}
\includegraphics{HW3_files/figure-latex/unnamed-chunk-18-7.pdf}
\includegraphics{HW3_files/figure-latex/unnamed-chunk-18-8.pdf}
\includegraphics{HW3_files/figure-latex/unnamed-chunk-18-9.pdf}
\includegraphics{HW3_files/figure-latex/unnamed-chunk-18-10.pdf}
\includegraphics{HW3_files/figure-latex/unnamed-chunk-18-11.pdf}
\includegraphics{HW3_files/figure-latex/unnamed-chunk-18-12.pdf}
\includegraphics{HW3_files/figure-latex/unnamed-chunk-18-13.pdf}
\includegraphics{HW3_files/figure-latex/unnamed-chunk-18-14.pdf}
\includegraphics{HW3_files/figure-latex/unnamed-chunk-18-15.pdf}
\includegraphics{HW3_files/figure-latex/unnamed-chunk-18-16.pdf}
\includegraphics{HW3_files/figure-latex/unnamed-chunk-18-17.pdf}
\includegraphics{HW3_files/figure-latex/unnamed-chunk-18-18.pdf}
\includegraphics{HW3_files/figure-latex/unnamed-chunk-18-19.pdf}
\includegraphics{HW3_files/figure-latex/unnamed-chunk-18-20.pdf} *
Evolution for the best solution in case \(K = 5\)

\begin{Shaded}
\begin{Highlighting}[]
\NormalTok{best.loss.seq.df <-}\StringTok{ }\KeywordTok{data.frame}\NormalTok{(}\DataTypeTok{loss=}\NormalTok{results[[}\DecValTok{1}\NormalTok{]]}\OperatorTok{$}\NormalTok{best.loss.seq,}\DataTypeTok{iter=}\KeywordTok{seq}\NormalTok{(}\DecValTok{1}\NormalTok{,}\KeywordTok{length}\NormalTok{(results[[}\DecValTok{1}\NormalTok{]]}\OperatorTok{$}\NormalTok{best.loss.seq)))}
\KeywordTok{ggplot}\NormalTok{(best.loss.seq.df,}\KeywordTok{aes}\NormalTok{(}\DataTypeTok{x=}\NormalTok{iter,}\DataTypeTok{y=}\NormalTok{loss)) }\OperatorTok{+}\StringTok{ }
\StringTok{  }\KeywordTok{geom_line}\NormalTok{() }\OperatorTok{+}\StringTok{ }\KeywordTok{geom_point}\NormalTok{() }\OperatorTok{+}\StringTok{ }\KeywordTok{ggtitle}\NormalTok{(}\DataTypeTok{label =} \StringTok{'K = 5'}\NormalTok{)}
\end{Highlighting}
\end{Shaded}

\includegraphics{HW3_files/figure-latex/unnamed-chunk-19-1.pdf} *
Evolution for the best solution in case \(K = 10\)

\begin{Shaded}
\begin{Highlighting}[]
\NormalTok{best.loss.seq.df <-}\StringTok{ }\KeywordTok{data.frame}\NormalTok{(}\DataTypeTok{loss=}\NormalTok{results[[}\DecValTok{2}\NormalTok{]]}\OperatorTok{$}\NormalTok{best.loss.seq,}\DataTypeTok{iter=}\KeywordTok{seq}\NormalTok{(}\DecValTok{1}\NormalTok{,}\KeywordTok{length}\NormalTok{(results[[}\DecValTok{2}\NormalTok{]]}\OperatorTok{$}\NormalTok{best.loss.seq)))}
\KeywordTok{ggplot}\NormalTok{(best.loss.seq.df,}\KeywordTok{aes}\NormalTok{(}\DataTypeTok{x=}\NormalTok{iter,}\DataTypeTok{y=}\NormalTok{loss)) }\OperatorTok{+}\StringTok{ }
\StringTok{  }\KeywordTok{geom_line}\NormalTok{() }\OperatorTok{+}\StringTok{ }\KeywordTok{geom_point}\NormalTok{() }\OperatorTok{+}\StringTok{ }\KeywordTok{ggtitle}\NormalTok{(}\DataTypeTok{label =} \StringTok{'K = 10'}\NormalTok{)}
\end{Highlighting}
\end{Shaded}

\includegraphics{HW3_files/figure-latex/unnamed-chunk-20-1.pdf} *
Evolution for the best solution in case \(K = 20\)

\begin{Shaded}
\begin{Highlighting}[]
\NormalTok{best.loss.seq.df <-}\StringTok{ }\KeywordTok{data.frame}\NormalTok{(}\DataTypeTok{loss=}\NormalTok{results[[}\DecValTok{3}\NormalTok{]]}\OperatorTok{$}\NormalTok{best.loss.seq,}\DataTypeTok{iter=}\KeywordTok{seq}\NormalTok{(}\DecValTok{1}\NormalTok{,}\KeywordTok{length}\NormalTok{(results[[}\DecValTok{3}\NormalTok{]]}\OperatorTok{$}\NormalTok{best.loss.seq)))}
\KeywordTok{ggplot}\NormalTok{(best.loss.seq.df,}\KeywordTok{aes}\NormalTok{(}\DataTypeTok{x=}\NormalTok{iter,}\DataTypeTok{y=}\NormalTok{loss)) }\OperatorTok{+}\StringTok{ }
\StringTok{  }\KeywordTok{geom_line}\NormalTok{() }\OperatorTok{+}\StringTok{ }\KeywordTok{geom_point}\NormalTok{() }\OperatorTok{+}\StringTok{ }\KeywordTok{ggtitle}\NormalTok{(}\DataTypeTok{label =} \StringTok{'K = 20'}\NormalTok{)}
\end{Highlighting}
\end{Shaded}

\includegraphics{HW3_files/figure-latex/unnamed-chunk-21-1.pdf} *
distribution of terminal loss function values in case \(K = 5\)

\begin{Shaded}
\begin{Highlighting}[]
\NormalTok{terminal.losses.df <-}\StringTok{ }\KeywordTok{data.frame}\NormalTok{(}\DataTypeTok{terminal.losses=}\NormalTok{results[[}\DecValTok{1}\NormalTok{]]}\OperatorTok{$}\NormalTok{terminal.losses)}
\KeywordTok{ggplot}\NormalTok{(terminal.losses.df,}\KeywordTok{aes}\NormalTok{(}\DataTypeTok{x=}\NormalTok{terminal.losses)) }\OperatorTok{+}\StringTok{ }
\StringTok{  }\KeywordTok{geom_density}\NormalTok{() }\OperatorTok{+}\StringTok{ }\KeywordTok{ggtitle}\NormalTok{(}\DataTypeTok{label =} \StringTok{'K = 5'}\NormalTok{)}
\end{Highlighting}
\end{Shaded}

\includegraphics{HW3_files/figure-latex/unnamed-chunk-22-1.pdf} *
distribution of terminal loss function values in case \(K = 10\)

\begin{Shaded}
\begin{Highlighting}[]
\NormalTok{terminal.losses.df <-}\StringTok{ }\KeywordTok{data.frame}\NormalTok{(}\DataTypeTok{terminal.losses=}\NormalTok{results[[}\DecValTok{2}\NormalTok{]]}\OperatorTok{$}\NormalTok{terminal.losses)}
\KeywordTok{ggplot}\NormalTok{(terminal.losses.df,}\KeywordTok{aes}\NormalTok{(}\DataTypeTok{x=}\NormalTok{terminal.losses)) }\OperatorTok{+}\StringTok{ }
\StringTok{  }\KeywordTok{geom_density}\NormalTok{() }\OperatorTok{+}\StringTok{ }\KeywordTok{ggtitle}\NormalTok{(}\DataTypeTok{label =} \StringTok{'K = 10'}\NormalTok{)}
\end{Highlighting}
\end{Shaded}

\includegraphics{HW3_files/figure-latex/unnamed-chunk-23-1.pdf} *
distribution of terminal loss function values in case \(K = 20\)

\begin{Shaded}
\begin{Highlighting}[]
\NormalTok{terminal.losses.df <-}\StringTok{ }\KeywordTok{data.frame}\NormalTok{(}\DataTypeTok{terminal.losses=}\NormalTok{results[[}\DecValTok{3}\NormalTok{]]}\OperatorTok{$}\NormalTok{terminal.losses)}
\KeywordTok{ggplot}\NormalTok{(terminal.losses.df,}\KeywordTok{aes}\NormalTok{(}\DataTypeTok{x=}\NormalTok{terminal.losses)) }\OperatorTok{+}\StringTok{ }
\StringTok{  }\KeywordTok{geom_density}\NormalTok{() }\OperatorTok{+}\StringTok{ }\KeywordTok{ggtitle}\NormalTok{(}\DataTypeTok{label =} \StringTok{'K = 20'}\NormalTok{)}
\end{Highlighting}
\end{Shaded}

\includegraphics{HW3_files/figure-latex/unnamed-chunk-24-1.pdf}

\section{2. Problem 2: Finite-state Hidden Markov models
(HMMs)}\label{problem-2-finite-state-hidden-markov-models-hmms}

(Continued from the problem on Markov chains from the previous
homework.)

Suppose now that we do not observe the state \(S_t\) of the Markov
chain. Instead, at time \(t\) we observe \(Y_t\). \(Y_t\) can be
anything: integers, reals, vectors, images. The only condition is that
the probability distribution of \(Y_t\) depends only on \(S_t\) (and not
\(\text{e.g.}\) on \(S_{t-1}\)). Write this as \(P_Y (Y_t|S_t)\), with
\(P_Y (\cdot |S_t = i)\) giving the probability (or probability density)
of \(Y_t\) given \(S_t = i\) (for simplicity, we let this same
probability hold for all \(t\)). Also write
\(\mathbf{Y} = (Y_1, . . . Y_T )\).

Our HMM model defines a probability distribution over
\((\mathbf{S}, \mathbf{Y})\).

\begin{enumerate}
\def\labelenumi{\arabic{enumi}.}
\tightlist
\item
  Write down
  \(P(S_1 = s_1, S_2 = s_2, \dots, S_T = s_T , Y_1 = y_1, \dots, Y_T = y_t)\)
  in terms of \(\pi^1\), \(A\) and \(P_Y\).
\end{enumerate}

For the earlier `Markov chain' problem, we could quite efficiently
calculate \(\pi_i^t = P(S_t = i)\), the marginal prior probability of
\(S_t\). We will now calculate the marginal posterior probabilities
\(P(S_t = i|\mathbf{Y})\). Look at the scanned notes for the Kalman
filter for reference.

From now onwards, we will fix the values of \(\mathbf{Y}\), since these
are our observations. Then define an \(N \times 1\) vector \(B^t\), with
\(B^t_i = P(Y_t = y_t|S_t = i)\).

\begin{enumerate}
\def\labelenumi{\arabic{enumi}.}
\tightlist
\item
  Is this a \(probability\) vector (i.e.~is it a nonnegative, adding up
  to 1)?
\end{enumerate}

Define \(\alpha\) and \(\beta\) messages:
\[\alpha_i^t := P(S_t = i, Y_1 = y_1, \dots, Y_t = y_t) = P(S_t = i, \mathbf{Y}_{1:t})\]
\[\beta_i^t := P(Y_{t+1} = y_{t+1}, \dots, Y_T = y_T | S_t = i) = P(\mathbf{Y}_{t+1:T} | S_t = i)\]
Above \(:=\) means `which we define to be equal to', or `which we will
call'.

\begin{enumerate}
\def\labelenumi{\arabic{enumi}.}
\setcounter{enumi}{1}
\item
  \(\alpha^t\) and \(\beta^t\) are both \(N \times 1\) vectors. Are
  these vectors \(probability\) vectors?
\item
  Write \(P(S_t = i | \mathbf{Y})\) in terms of \(\alpha^t\) and
  \(\beta^t\). Include the normalization term (i.e summation of both
  sides over \(i\) must give 1). In the scanned notes (e.g.~top
  paragraph of page 2), we ignore the normalization constant, but we can
  easily calculate it since we know that probabilities must sum or
  integrate to 1. Use matrix (or vector) notation. You might have to use
  a transpose (e.g. \((\alpha^t)^T\)). Also, define \(\mathbf{1}\) as an
  \(N \times 1\) vector of ones, and note that
  \(\sum_{i=1}^N a_i^t = \mathbf{1}^Ta^t\). Hint: First write it
  explicitly with summations (compare with the Kalman filter).
\item
  Write \(P(S_t = i, S_{t+1} = j|\mathbf{Y})\) in terms of \(\alpha^t\),
  \(\beta^t\), \(A\) and \(B\). Include the normalization term and use
  matrix (or vector) notation.
\item
  Write \(\alpha^t\) as a function of \(\alpha^{t-1}\), \(A\) and \(B\).
  Use matrix notation. An operation you'll need here is the element-wise
  product of two vectors. In R this is easy, just write \(V_1 \ast V_2\)
  for two vectors \(V_1\) and \(V_2\). In matrix notation, the simplest
  way to write this is as \(diag(V_2) \cdot V_1\), where \(diag(V_2)\)
  is an \(N \times N\) matrix whose diagonal is \(V_2\) and whose other
  elements are 0. Verify that \(diag(V_2) \cdot V_1\) is a vector whose
  \(i\)th element is the product of the \(i\)th elements of \(V_1\) and
  \(V_2\).
\item
  Write \(\beta^t\) as a function of \(\beta^{t+1}\), \(A\) and \(B\).
  Use matrix notation.
\item
  How will you calculate the first \(\alpha\) and \(\beta\) at the
  beginning of the forward and backward pass?
\item
  Hopefully, you can now see a dynamic programming algorithm that
  sequentially calculates the \(\alpha^{t\prime}\)s, and then the
  \(\beta^{t\prime}\)s. These can then be combined to calculate
  \(P(S_t|\mathbf{Y})\) for any \(t\). The overall algorithm is called
  the Baum-Welch algorithm. Write down the cost in terms of \(T\) and
  \(N\).
\item
  Imagine instead we wanted to calculate the most like sequence of
  states (rather than the marginal probabilities). One approach is to
  set \(S_t = \arg \max P(S_t|\mathbf{Y})\) for all \(t\). Why is this a
  bad idea?
\end{enumerate}

\textbf{Solution:}

\begin{enumerate}
\def\labelenumi{\arabic{enumi}.}
\item
  The joint probability is given by \[\begin{align}
  P(S_1 = s_1, \dots, S_T = s_T , Y_1 = y_1, \dots, Y_T = y_t) & = P(Y_1 = y_1,\dots,Y_T = y_T|S_1=s_1,\dots,S_T=s_T)P(S_1 = s_1,\dots,S_T=s_T) \\
  & = \prod_{t=1}^T P_Y(Y_t=y_t|S_t=s_t) \cdot P(S_1 = s_1,\dots,S_T=s_T) \\
  & = \prod_{t=1}^T P_Y(Y_t=y_t|S_t=s_t) \cdot P(S_1=s_1)\cdot \prod_{i=2}^T P(S_i=s_i|S_{i-1}=s_{i-1}) \\
  & = \prod_{t=1}^T P_Y(Y_t=y_t|S_t=s_t) \cdot \pi^1_{s_1}\cdot \prod_{i=2}^T A_{s_{i-1},s_i}
  \end{align}\]
\item
  No, \(B^t\) is not a probability vector.
\item
  \(\alpha^t\) is not a probability vector since
  \(\sum_{i} \alpha_i^t = P(\mathbf{Y}_{1:t})\). \(\beta^t\) is not a
  probability vector.
\item
  The probability is given by \[\begin{align}
  P(S_t = i | \mathbf{Y}) & = \frac{P(S_t=i,\mathbf{Y})}{P(\mathbf{Y})} \\
  & = \frac{P(S_t = i,\mathbf{Y}_{1:t})P(\mathbf{Y}_{t+1:T}|S_t=i)}{(\alpha^t)^\top\beta^t} \\
  & = \frac{\alpha^t_i\beta^t_i}{(\alpha^t)^\top\beta^t}
  \end{align}\]
\item
  The probability is given by \[\begin{align}
  P(S_t = i, S_{t+1} = j | \mathbf{Y}) & = \frac{P(S_t = i, S_{t+1} = j, \mathbf{Y})}{P(\mathbf{Y})} \\
  & = \frac{P(S_t=i,\mathbf{Y}_{1:t})P(S_{t+1}=j,\mathbf{Y}_{t+1:T}|S_t=i,\mathbf{Y}_{1:t})}{(\alpha^t)^\top\beta^t} \\
  & = \frac{\alpha^t_i P(\mathbf{Y}_{t+1:T}|S_t=i,S_{t+1}=j,\mathbf{Y}_{1:t})P(S_{t+1}=j|S_t=i,\mathbf{Y}_{1:t})}{(\alpha^t)^\top\beta^t} \\
  & = \frac{\alpha^t_i P(\mathbf{Y}_{t+2:T}|S_t=i,S_{t+1}=j,\mathbf{Y}_{1:t+1})P(Y_{t+1} = y_{t+1}|S_t = i, S_{t+1}=j,\mathbf{Y}_{1:t})P(S_{t+1}=j|S_t=i)}{(\alpha^t)^\top\beta^t} \\
  & = \frac{\alpha^t_i P(\mathbf{Y}_{t+2:T}|S_{t+1}=j)P(Y_{t+1} = y_{t+1}| S_{t+1}=j)P(S_{t+1}=j|S_t=i)}{(\alpha^t)^\top\beta^t} \\
  & = \frac{\alpha^t_i \beta^{t+1}_jB^{t+1}_jA_{ij}}{(\alpha^t)^\top\beta^t}
  \end{align}\]
\end{enumerate}

Dividing \(\alpha^t\) and \(\beta^t\) by their length to get the
normalized forward and backward messaging vectors, i.e.,
\[\tilde{\alpha^t_i} = \frac{\alpha^t_i}{\mathbf{1}^\top\alpha^t}, \quad \tilde{\beta^t_i} = \frac{\beta^t_i}{\mathbf{1}^\top\beta^t}\]
Replacing \(\alpha^t_i\) and \(\beta^t_i\) with their normalized
versions in \(P(S_t = i, S_{t+1} = j | \mathbf{Y})\) can render the
results of this question.

\begin{enumerate}
\def\labelenumi{\arabic{enumi}.}
\setcounter{enumi}{4}
\tightlist
\item
  The relational formaula is derived as follows \[\begin{align}
  \alpha^t_i & = P(S_t=i,\mathbf{Y}_{1:t}) \\
  & = \sum_j P(S_t = i, S_{t-1} = j, \mathbf{Y}_{1:t}) \\
  & = \sum_j P(S_{t-1} = j, \mathbf{Y}_{1:t-1})P(S_t = i, Y_t = y_t | S_{t-1} = j, \mathbf{Y}_{1:t-1}) \\
  & = \sum_j \alpha^{t-1}_j P(Y_t = y_t | S_t = i, S_{t-1} = j, \mathbf{Y}_{1:t-1})P(S_t=i | S_{t-1} = j, \mathbf{Y}_{1:t-1}) \\
  & = \sum_j \alpha^{t-1}_j P(Y_t = y_t | S_t = i)P(S_t=i | S_{t-1} = j) \\
  & = \sum_j \alpha^{t-1}_j B^t_i A_{ji}
  \end{align}\]
\end{enumerate}

The matrix form is given by
\[\alpha^t = diag(B^t)\left(A^\top \alpha^{t-1}\right)\]

\begin{enumerate}
\def\labelenumi{\arabic{enumi}.}
\setcounter{enumi}{5}
\tightlist
\item
  The relational formula is derived as follows \[\begin{align}
  \beta^t_i & = P(\mathbf{Y}_{t+1:T}|S_t = i) \\
  & = \sum_j P(S_{t+1}=j,\mathbf{Y}_{t+1:T}|S_t = i) \\
  & = \sum_j P(S_{t+1}=j|S_t = i) P(\mathbf{Y}_{t+1:T} | S_t=i, S_{t+1} = j) \\
  & = \sum_j P(S_{t+1}=j|S_t = i) P(\mathbf{Y}_{t+2:T} | Y_{t+1} = y_{t+1}, S_t=i, S_{t+1} = j) P(Y_{t+1} = y_{t+1}| S_t=i, S_{t+1} = j) \\
  & = \sum_j P(S_{t+1}=j|S_t = i) P(\mathbf{Y}_{t+2:T} | S_{t+1} = j) P(Y_{t+1} = y_{t+1}| S_{t+1} = j) \\
  & = \sum_j A_{ij} \beta^{t+1}_j B^{t+1}_j
  \end{align}\]
\end{enumerate}

Therefore, the matrix form is given by
\[\beta^t = A\left(diag(B^{t+1})\beta^{t+1}\right)\]

\begin{enumerate}
\def\labelenumi{\arabic{enumi}.}
\setcounter{enumi}{6}
\item
  The initial values are given by
  \[\alpha^1_i = P(Y_1=y_1|S_1=i)P(S_1=i) = B^1_i\pi^1_i\]
  \[\beta^T_i = 1\]
\item
  The cost of Baum-Welch algorithm is \(2(N+N^2)T + 2 + N = O(TN^2)\).
\item
  It is a bad idea since the complexity is \(O(T^3(D+d)^3)\). If the
  size of observations is large, the time is consumption is huge and the
  system will crack down.
\end{enumerate}


\end{document}
